%----------------------------------------------------------------------------------------
%	PACKAGES AND OTHER DOCUMENT CONFIGURATIONS
%----------------------------------------------------------------------------------------

\documentclass[a4paper,10pt]{article}

% colours
\usepackage[usenames,dvipsnames]{xcolor}
% local margin changes using \adjustwidth
\usepackage{changepage}
% ???
\usepackage{comment}

% margins
\usepackage[bottom=0.9cm,top=0.9cm,left=1.5cm,right=1.5cm]{geometry}

% no paragraph indents
\setlength{\parindent}{0pt}

\PassOptionsToPackage{hyphens}{url}
\usepackage{hyperref}
\hypersetup{%
	colorlinks=true,% hyperlinks will be coloured
	urlcolor = blue,
	linkcolor = blue,
	anchorcolor = blue,
	pdfborderstyle={/S/U/W 1}% border style will be underline of width 1pt
}

% customise \section
\usepackage{titlesec}
\titleformat{\section}{\large\scshape\raggedright}{}{0em}{}[\titlerule]
\titlespacing{\section}{0pt}{3pt}{5pt}

%%% Tightly-spaced list
\newcommand{\squishlist}{
	\begin{list}{$\bullet$}
	{
		\setlength{\itemsep}{0pt}
		\setlength{\parsep}{0pt}
		\setlength{\topsep}{0pt}
		\setlength{\partopsep}{0pt}
		\setlength{\leftmargin}{2em}
		\setlength{\labelwidth}{1.5em}
		\setlength{\labelsep}{0.5em}
	}
}

\newcommand{\squishend}{\end{list}}

\begin{document}

\pagestyle{empty} % Removes page numbering

% ===========
% NAME - MUY IMPORTANTE
% ===========
\par{\centering{\textbf{\large Artur K. Lidtke}}\par\par}

% ===========
% SUMMARY
% ===========
\section{Summary}

\hspace{1cm}
\begin{minipage}{\textwidth}
\begin{adjustwidth}{}{2cm} % reduce the right margin by 1 cm
% underwater vehicle design
	Self-motivated researcher with background in scientific
		software development, experimental
		and numerical fluid dynamics, and applied R\&D.
	Keen on applying knowledge to practical engineering problems,
		characterised by a can-do attitude and commitment to his work.
\end{adjustwidth}
\end{minipage}

% ===========
% PERSONAL INFO
% ===========
\section{Personal information}

\begin{minipage}{\textwidth}
\begin{adjustwidth}{}{1cm} % reduce the right margin by 1 cm

%\begin{tabular}{l l l}
\textbf{Address:} XXX\\
\textbf{Telephone:} XXX \hspace{2cm} \textbf{Email:} artur.lidtke@gmail.com
%\end{tabular}

\end{adjustwidth}
\end{minipage}

%----------------------------------------------------------------------------------------
%	WORK EXPERIENCE
%----------------------------------------------------------------------------------------

\section{Work and professional experience}

\begin{minipage}{\textwidth}
\begin{adjustwidth}{}{1cm}

%\renewcommand{\arraystretch}{0.8}
\begin{tabular}{p{3.5cm} p{12.9cm}}

\textbf{Employment} \\
\\

%---------------
\textsc{Sep 2018 - Present} & \textsc{Senior Researcher in Computational Fluid Dynamics RnD Group} \\
							& Maritime Research Institute Netherlands (MARIN), Wageningen, The Netherlands \vspace{0.1cm} \\
%
\multicolumn{2}{l}{
\hspace{1cm}\begin{minipage}[t]{\textwidth}
\begin{adjustwidth}{}{2cm}
\squishlist
	\item Development of new and improvement of existing functionality of the in-house computational fluid dynamic code, ReFRESCO (Fortran 95 \& 03), and associated automated test framework.
	\item Management of on-going research projects, identification of new opportunities and external funding.
	\item Superivision of PhD- and MSc-level students.
	\item Detailed validation of developed models and implementation of automated test suites.
	\item Data analysis and automation (Python, bash).
	\item Development and application of machine learning methods to fluid dynamic problems.
%	\item Wind tunnel test preparation, measurements and data analysis.
\squishend
\end{adjustwidth}
\end{minipage}
} \\
\\

%---------------
\textsc{Jan 2016 - Aug 2018} & \textsc{Research Fellow in Maritime Robotics - EU BRIDGES consortium} \\
							& Fluid Structure Interaction Group, University of Southampton, United Kingdom \vspace{0.1cm} \\
%
\multicolumn{2}{l}{
\hspace{1cm}\begin{minipage}[t]{\textwidth}
\begin{adjustwidth}{}{2cm}
\squishlist
	\item Undertook hydrodynamic and mechanical design of two autonomous underwater gliders.
	\item Used computational and experimental fluid dynamic analysis for performance evaluation of underwater vehicle hulls, rudders and thrusters.
	\item Led the hydrodynamic design work package liaising with industrial and academic partners to establish project timelines and prioritise design activities on system and sub-system level.
	\item Designed, built, tested and used wind tunnel dynamometry systems for precision measurements.
	\item Wrote software for propeller design, AUV mission simulation, data analysis (Python).
\squishend
\end{adjustwidth}
\end{minipage}
} \\
\\

%---------------
\textsc{Oct 2013 - Jun 2017} & \textsc{Doctoral Researcher in Marine Hydrodynamics} \\%, Fluid Structure Interactions Research Group} \\
							& Fluid Structure Interaction Group, University of Southampton, United Kingdom \vspace{0.1cm} \\
%
\multicolumn{2}{l}{
\hspace{1cm}\begin{minipage}[t]{\textwidth}
\begin{adjustwidth}{}{2cm}
\squishlist
	\item Carried out development of acoustic tools (Ffowcs Williams-Hawkings analogy) and numerical models for cavitation prediction inside the OpenFOAM package (C++).
	\item Applied both methodologies to study the hydroacoustics of marine propellers and hydrofoils.
	\item Developed data analysis and CFD automation tools (Python and C++).
%	\item Engaged in undergraduate teaching of Python programming, hydro- and aero-dynamic design.
\squishend
\end{adjustwidth}
\end{minipage}
} \\

\\
\multicolumn{2}{l}{\textbf{Consultancy and short-term projects}} \\
\\

\textsc{Nov 2018 - Apr 2019} & Part time consultant on AUV propulsion (self employed) \\
%
\multicolumn{2}{l}{
\hspace{1cm}\begin{minipage}[t]{\textwidth}
\begin{adjustwidth}{}{2cm}
\squishlist
	\item Established project plan and priorities within customer budget.
	\item Performed design analysis of ducted propulsion systems of two small AUV's.
	\item Proposed alternative designs to improve propulsion and help the vehicles meet target speeds.
\squishend
\end{adjustwidth}
\end{minipage}
} \\
\\

\begin{comment}
\textsc{Dec 2014, Dec 2015, Oct 2016} & Part of Fluid Structure Interaction Group teams
		participating in international ship hydrodynamic workshops
		(SIMMAN '14, Tokyo '15, LR Southampton '16) \vspace{0.1cm} \\
%
\multicolumn{2}{l}{
\hspace{1cm}\begin{minipage}[t]{\textwidth}
\begin{adjustwidth}{}{2cm}
\squishlist
	\item Coordinated efforts of several contributors and organised workshop submissions.
	\item Performed numerical simulations of flows past ship hulls and propellers.
\squishend
\end{adjustwidth}
\end{minipage}
} \\
\\
\end{comment}

\textsc{Apr 2015} & Consultancy work for Longitude Engineering, Southampton \vspace{0.1cm} \\
%
\multicolumn{2}{l}{
\hspace{1cm}\begin{minipage}[t]{\textwidth}
\begin{adjustwidth}{}{2cm}
\squishlist
	\item Liaised with the client to come up with software requirements specification.
	\item Developed a bespoke numerical model in Simulink to describe manned submersible dynamics.
		% of a novel submersible design for naval applications.
	\item Used the model to perform controllability analysis and suggested improvements to the design.
	%\item Led the work package and managed the efforts of a second collaborator.
\squishend
\end{adjustwidth}
\end{minipage}
} \\


\\
\textbf{Internships} \\
\\

%---------------
\textsc{Jun - Sep 2013} & \textsc{Summer Intern at CJR Propulsion Limited} \\
						& 72 Quayside Road, SO18 1AD, Southampton, UK \vspace{0.1cm} \\
%
\multicolumn{2}{l}{
\hspace{1cm}\begin{minipage}[t]{\textwidth}
\begin{adjustwidth}{}{2cm}
\squishlist
	\item Developed CFD simulations and data analysis scripts for ship flow applications.
%	\item Applied OpenFOAM for hydrodynamic analysis of propellers and ship hull.
	\item Developed Python scripts for automated data analysis and improved design practices.
%	\item Engaged in mechanical design and gained experience in manufacturing (3D printing, casting, CNC).
%	\item Carried out sea trials of a motor yacht to assess noise and vibration of its propeller.
\squishend
\end{adjustwidth}
\end{minipage}
} \\
\\

%---------------
\textsc{Jun - Aug 2012} & \textsc{Research Intern at University of Southampton} \\
						& Fluid Structure Interaction Group, University of Southampton, United Kingdom \vspace{0.1cm} \\
%
\multicolumn{2}{l}{
\hspace{1cm}\begin{minipage}[t]{\textwidth}
\begin{adjustwidth}{}{2cm}
\squishlist
	\item Performed a technological, economic, and regulatory feasibility study looking at suitability of
		small modular nuclear reactors (SMRs) for propulsion of commercial ships.
	\item Developed bespoke techno-economic evaluation tools (C++ and Matlab).
	\item Worked with project partners to support a \pounds10M grant application to the UK Government.
\squishend
\end{adjustwidth}
\end{minipage}
} \\

\end{tabular}

\end{adjustwidth}
\end{minipage}


%----------------------------------------------------------------------------------------
%	SKILLS
%----------------------------------------------------------------------------------------
\section{Key skills}

\begin{minipage}{\textwidth}
\begin{adjustwidth}{}{1cm}

\textbf{Areas of specialisation and professional experience}
\squishlist
	\item Expertise in development of scientific and engineering software for a wide variety
		of applications using multiple programming languages and operating systems with
		specialisation in Python \& C++ on Linux.
	% \item Comprehensive understanding of engineering principles related to all aspects of marine
	% 	craft design with particular focus on unmanned underwater systems developed from work for the BRIDGES consortium.
	\item Deep understanding of marine hydrodynamics, with particular emphasis on
		performance prediction, hydroacoustics and ship wake flows obtained during PhD, employment, and part-time projects.
	%\item Experience in designing carrying out wind tunnel and towing tank experiments derived from
	%	testing of underwater vehicles, hydrofoils, ship hulls and moored systems.
\squishend

\textbf{Job-related skills}
\squishlist
	\item Problem solving skills in areas such as applied mathematics and physical system modelling
		derived from multiple projects centred around development of new models and methodologies.
	% \item Understanding of autonomous robot design and operation with focus on underwater robotics developed during employment, attendance of
	% 	conferences and training courses.
	% \item Experience in development of physical models of dynamical systems, surface and underwater vehicles.
	% \item Ability to design, build and utilise electro-mechanical systems obtained from work
	% 	on custom dynamometry rigs, soft robotic actuators, and hobby robotics.
	\item Outstanding learning skills and ability to quickly adapt and cope with novel problems.
	%\item Experience in applying computer programming and numerical analysis to solving wide variety of
	%	engineering problems, e.g. vehicle simulation,
	%	techno-economic evaluation, acoustic analysis and cavitation modelling
	%\item Comprehensive understanding of engineering principles related to all aspects of marine craft design
	%	gained throughout education and industrial experience
    \item Rich numerical modelling background with strong focus on marine CFD, particularly
        multiphase flows, turbulence and acoustics.
%    \item Exposure to fundamentals of Oceanography during work on autonomous underwater vehicles for scientific applications.
	\item Practical knowledge of common machine learning and reinforcement learning frameworks (scipy, pytroch, stable baselines).
	\item Ability to communicate with stakeholders, clients, and researchers in order to understand capability gaps in modelling tools and scope opporunities for future developments.
\squishend

\textbf{IT skills}
\squishlist
	\item Rich background in data analytics with focus on using Python in order to assimilate various streams of
		data to inform decisions and support quantitative scientific conclusions.
	\item Proficient programming skills in Python, C/C++ and Fortran; rudimentary experience in MatLab.
	\item Experience of software development for Windows and Linux operating systems.
	\item Ability to carry out test-driven development of complex and constantly evolving software.
	\item Version control using git and SVN for individual and collaborative software projects.
	% \item Experience in software development for Arduino, basic understanding of embedded programming and Robot Operating System (ROS).
	%\item Working knowledge of fluid dynamics, particularly in relation to marine applications
	%\item Elementary teaching experience obtained during group and individual tutoring on
	%	ship resistance and propulsion, informal guidance of new PhD students learning the OpenFOAM
	%	CFD package, as well as co-leading an OpenFOAM skills development workshop
	%\item Knowledge of some of the commonly used global optimisation and computer-aided decision making
	%	techniques and their applicability during an engineering product design
	%\item Ability to work as an independent member of a multidisciplinary team and produce deliverables contributing
	%	to the final outcome of the project and taking part of the responsibility for its realisation
\squishend

\textbf{Communication skills}
\squishlist
	\item Effective written and oral communication developed through writing of journal articles,
		project reports, and attendance of conferences.
	\item Experience in conveying own knowledge through undergraduate teaching, guidance of students.
		through individual and group projects, supporting fellow researchers in using OpenFOAM CFD package.
\squishend

\textbf{Organisation and managerial skills}
\squishlist
	\item Decision making and project management skills developed during tackling several parallel commitments with strict deadlines
		and commercial implications thereof.
	\item Ability to cooperate with and manage people developed during participating in several group
		projects, supervising intern and undergraduate project students, and acting as a~sailing instructor.
\squishend

\end{adjustwidth}
\end{minipage}

%----------------------------------------------------------------------------------------
%	EDUCATION
%----------------------------------------------------------------------------------------

\section{Education}

\begin{minipage}{\textwidth}
\begin{adjustwidth}{}{1cm}

\begin{tabular}{p{3.8cm} p{12.5cm}}

%& \textbf{Project title:} Predicting radiated noise of marine propellers using acoustic analogies
%	and hybrid Eulerian-Lagrangian cavitation models \\
%& \begin{tabular}{l p{10.2cm}}
%	Supervisors: 	& Prof. S. R. Turnock (Fluid Structure Interactions Research Group) \\
%					& Prof. V. F. Humphrey (Institute of Sound \& Vibration Research) \\
%	Topic:			& Use of Large Eddy Simulation, mass-transfer and Lagrangian cavitation models and
%						acoustic analogies for marine propeller noise characterisation. \\
%	Impact:			& Proved the concept and validated the acoustic analogy for modelling
%						cavitation noise. This fed into a new wave of research aiming at
%						reducing ship-induced noise pollution; applied the hybrid Eulerian-Lagrangian
%						methodology in a new context and identified a number of ways
%						for it to be improved. \\
%\end{tabular} \\

\textsc{Oct 2013 - Jun 2017} & \textsc{PhD in Marine Hydrodynamics} \\%, Fluid Structure Interactions Research Group} \\
							& Fluid Structure Interaction Group, University of Southampton, United Kingdom \\
\\

%---------------
\textsc{Oct 2009 - Jun 2013} & MEng (Hons) \textsc{Ship Science – Yacht \& Small Craft, 1$^{st}$ class} \\
& University of Southampton, Southampton, United Kingdom \vspace{0.1cm} \\
%
\multicolumn{2}{l}{
\hspace{1cm}\begin{minipage}[t]{\textwidth}
\begin{adjustwidth}{}{2cm}
\squishlist
	\item Naval architecture course with selected modules focused on marine craft hydrodynamic design.
	\item BEng project looked at implementing fuzzy-adaptive genetic algorithms to aid in the selection
		of optimum yacht characteristics at the initial design stage (Matlab).
	\item MEng group project focused on the development of a real-time dynamic model of an America's Cup
		catamaran to be used for crew training (Simulink).
		%; work fed experience into several subsequent projects
		%at the university and external organisations (e.g. Wolfson Unit MTIA)
\squishend
\end{adjustwidth}
\end{minipage}
} \\

%---------------
%\textsc{Sep 2006 - Jun 2009} & \textsc{II Liceum Ogolnoksztalcace, Walbrzych, Poland} \\
%& \textsc{Physics, Mathematics and Computer Science programme}
\end{tabular}

\end{adjustwidth}
\end{minipage}

%----------------------------------------------------------------------------------------
%	OTHER ACTIVITIES
%----------------------------------------------------------------------------------------

\section{Relevant extracurricular experience}

\begin{minipage}{\textwidth}
\begin{adjustwidth}{}{1cm}

{\renewcommand{\arraystretch}{1.2}
\begin{tabular}{l p{13.0cm}}
%
\textsc{Oct 2014 - Jun 2015}	& Member of University of Southampton's team working on
									Global Marine Technology Trends (GMTT) 2030 report
									in collaboration with Lloyd's Register and QinetiQ \\
%
\textsc{Nov 2013}				& Co-organiser and demonstrator at OpenFOAM workshop at Woods Hole Oceanographic Institution (WHOI) \\
%
\textsc{Oct 2013 - Present}		& Supporting OpenFOAM knowledge-base of the research group through
									local workshops and individual engagement with fellow students and researchers \\
%
\textsc{Oct 2013 - Sep 2018}		& Teaching and lab demonstrator at undergraduate modules - Ship Resistance and Propulsion,
									Renewable Energy from Environmental Flows, Offshore Engineering \\
%
%\textsc{Apr 2012 - Jun 2013}	& Voluntary member of Atlantic Challenge England \\
%
%\textsc{Oct 2012 - Mar 2013} 	& Voluntary for Hampshire and Wight Trust for Maritime Archaeology \\
%
\textsc{Jun - Sept 2011} 		& Voluntary deckhand and sailing instructor aboard sails training ship Fryderyk Chopin \\
%
%\textsc{Oct 2010 - May 2011} 	& Member of team which took 5th place in IBM University Business Challenge '11 finals \\
\end{tabular}
}

\end{adjustwidth}
\end{minipage}

%----------------------------------------------------------------------------------------
%	LANGUAGES
%----------------------------------------------------------------------------------------

\section{Languages spoken}

\begin{minipage}{\textwidth}
\begin{adjustwidth}{}{0cm}

\textsc{English} - Professional (C2), %
\textsc{Polish} - Native, %
\textsc{Dutch} - Fair (B2)  %
\textsc{German} - Basic (A1),

\end{adjustwidth}
\end{minipage}

%----------------------------------------------------------------------------------------
%	OTHER STUFF
%----------------------------------------------------------------------------------------

\section{Other qualifications and relevant information}

\begin{minipage}{\textwidth}
\begin{adjustwidth}{}{0cm}

ISSA Leisure Yacht Skipper,
RYA VHF radio operator,
RYA First Aid Course,
Padi Advanced Open Water Diver
%, US visa B-category, Driving licence B Category

\end{adjustwidth}
\end{minipage}

\newpage

%----------------------------------------------------------------------------------------
%	TEACHING
%----------------------------------------------------------------------------------------

% \section{Teaching experience}
%
% \begin{minipage}{\textwidth}
% \begin{adjustwidth}{}{1cm}
%
% {\renewcommand{\arraystretch}{1.2}
% \begin{tabular}{l p{11.0cm}}
% \textsc{Oct 2017 - Sep 2019} & Teaching assistant in Renewable Energy from Environmental Flows undergraduate module \\
% %
% \textsc{Jun 2017 - Present} & Published \href{https://github.com/UnnamedMoose/BasicOpenFOAMProgrammingTutorials}{
% 		\underline{\smash{open-source training materials for OpenFOAM developers}}} \\
% %
% \textsc{Oct 2016 - Sep 2019} & Co-organiser and demonstrator during towing tank experiments for undergraduate modules \\
% %
% \textsc{Jan - Jun 2016} & Co-supervisor of an undergraduate exchange student looking at numerical modelling
% 							of turbulent transition \\
% %
% \textsc{Jan 2017 - Sep 2019} & Support of several PhD students in terms of learning the OpenFOAM CFD toolbox
% 								and diagnosing problems in their simulations \\
% %
% \textsc{Nov 2014}				& Demonstrator at an OpenFOAM workshop organised by the
% 									National Oceanographic Institution (NOC) and
% 									Woods Hole Oceanographic Institution (WHOI) \\
% %
% \textsc{Oct 2013 - Sep 2019} & Teaching assistant in Advances in Ship Resistance and Propulsion undergraduate module \\
% %
% \textsc{Oct 2013 - Sep 2019} & Teaching assistant in Advances in Ship Resistance and Propulsion
% 								undergraduate module \\
% \end{tabular}
% }
%
% \end{adjustwidth}
% \end{minipage}

%----------------------------------------------------------------------------------------
%	STUDENT PROJECTS
%----------------------------------------------------------------------------------------

% TODO find the titles and update.

% \section{MSc projects}
%
% \begin{minipage}{\textwidth}
% \begin{adjustwidth}{}{1cm}
%
% with Gabe:
%     Camilla Blakesley
%     de luca a. (adl2n17)
%     wang t. (tw8n17)
%     vignon l. (lv1n17)
%     gonzalez g.t. (gtg1u17)
%
% with Stephen:
%     Sebastien Lemaire
%
% at MARIN:
%     Eduardo Katsuno
%     Maik Dunkelberg
%     Atillio Scussel
%     (?) Tzu-Yao Huang - "Numerical Investigation of Bubble Entrapment with Tip-vortex Cavitation"
%
% PhD students
%   Chiara Wielgosz
%   Tzu-Yao Huang
%   Sankalp Jenna
%
% \end{adjustwidth}
% \end{minipage}


%----------------------------------------------------------------------------------------
\end{document}
