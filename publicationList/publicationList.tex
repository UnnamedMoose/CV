\documentclass[a4paper,10pt]{article}

% local margin changes using \adjustwidth
\usepackage{changepage}

% margins
\usepackage[bottom=1.5cm,top=2cm,left=1.5cm,right=2cm]{geometry}

% no paragraph indents
\setlength{\parindent}{0pt}

% links
\usepackage{hyperref}
\usepackage{textpos}
\usepackage{etoolbox}
\appto\UrlBreaks{\do\-} % break urls after these characters
\appto\UrlBreaks{\do\o}

\begin{document}

\pagestyle{empty} % Removes page numbering

\begin{textblock*}{100mm}(.70\textwidth,-1cm)
Last updated on \today
\end{textblock*}

%=====================
\section{Publication list}

%====
\subsection{Peer-reviewed journal articles}
%
\begin{itemize}
%
\item Lidtke, A. K., Lloyd, T. P., Lafeber, F-H, Bosschers, J (2022),
    "Predicting cavitating propeller noise in off-design conditions using scale-resolving CFD simulations",
    Ocean Engineering, 254
    \cite{lidtke_predicting_2022}
    \\ (10.1016/j.oceaneng.2022.111176)
    \\ \url{https://www.sciencedirect.com/science/article/pii/S0029801822005820?via\%3Dihub}
    \\ \url{https://www.researchgate.net/publication/360188763_Predicting_cavitating_propeller_noise_in_off-design_conditions_using_scale-resolving_CFD_simulations}

%
\item Lidtke, A. K., Klapwijk, M., Lloyd, T. P. (2021),
    "Scale-Resolving Simulations of a Circular Cylinder Subjected to Low Mach Number Turbulent Inflow",
    Journal of Marine Science and Engineering, 9(11)
    \cite{Lidtke2021b}
    \\ (doi:10.3390/jmse9111274)
    \\ \url{https://www.mdpi.com/2077-1312/9/11/1274}
    \\ \url{https://www.researchgate.net/publication/356282116_Scale-Resolving_Simulations_of_a_Circular_Cylinder_Subjected_to_Low_Mach_Number_Turbulent_Inflowm}
%
\item Katsuno, E. T., Lidtke, A. K., Bulent, D., Rijpkema, D., Dantas, J. L. D., Vaz, G. (2021)
    "Parameter uncertainty, Discretization uncertainty, Uncertainty quantification analysis, Sobol indices, Transitional flow, CFD",
    Computers \& Fluids, 230
    \cite{Katsuno2021}
    \\ (doi:10.1016/j.compfluid.2021.105129)
    \\ \url{https://www.sciencedirect.com/science/article/pii/S004579302100270X}
    \\ \url{https://www.researchgate.net/publication/354196190_Estimating_parameter_and_discretization_uncertainties_using_a_laminar-turbulent_transition_model}
%
\item Lidtke, A. K., Turnock, S. R. and Downes, J. (2021),
    "End-to-end efficiency quantification of an autonomous underwater vehicle propulsion system",
    Ocean Engineering, 234
    \cite{Lidtke2021a}
    \\ (doi:10.1016/j.oceaneng.2021.109223)
    \\ \url{https://www.sciencedirect.com/science/article/pii/S002980182100651X}
    \\ \url{https://www.researchgate.net/publication/352038822_End-to-end_efficiency_quantification_of_an_autonomous_underwater_vehicle_propulsion_system}
%
\item Lidtke, A. K., Turnock, S. R. and Downes, J. (2019),
	"Characterizing Influence of Transition to Turbulence on the Propulsive Performance of Underwater Gliders",
	Journal of Ship Research, 63, pp. 1-8
	\cite{Lidtke2019}
	\\ (doi:10.5957/JOSR.09180050)
	\\ \url{https://eprints.soton.ac.uk/430450/}
	\\ \url{https://www.ingentaconnect.com/contentone/sname/jsr/pre-prints/content-josr_09180050}
%
\item Higgens, A. D., Lidtke, A. K., Joseph, P. F., Turnock, S. R. (2020),
	"Investigation into the Tip-Gap Flow and Its Influence on Ducted Propeller Tip-Gap Noise Using Acoustic Analogies",
	Journal of Ship Research, 63, pp. 1-16
	\cite{Higgens2020}
	\\ (doi: 10.5957/JOSR.09180086)
	\\ \url{https://www.ingentaconnect.com/content/sname/jsr/pre-prints/content-josr_09180086}
	\\ \url{https://www.researchgate.net/publication/335773945_Investigation_into_the_Tip-Gap_Flow_and_Its_Influence_on_Ducted_Propeller_Tip-Gap_Noise_Using_Acoustic_Analogies}
%
\item Giorgio-Serchi, F., Lidtke, A. K. and Weymouth, G. D.,
	"A Soft Aquatic Actuator for Unsteady Peak Power Amplification",
	IEEE/ASME Transactions on Mechatronics, 23, pp. 2968-2973
	\cite{GiorgioSerchi2018}
	\\ (doi:10.1109/TMECH.2018.2873253)
	\\ \url{https://eprints.soton.ac.uk/423854}
	\\ \url{https://ieeexplore.ieee.org/document/8477111}
%
\item Lidtke, A. K., Turnock, S. R. and Downes, J. (2017),
	"Hydrodynamic Design of Underwater Gliders Using k-k$_L$-$\omega$ RANS Transition Model",
	Journal of Oceanic Engineering, 43, pp. 356-368
	\cite{Lidtke2017}
	\\ (doi:10.1109/JOE.2017.2733778),
	\\ \url{https://eprints.soton.ac.uk/412548/}
%
\item Lidtke, A. K., Turnock, S. R. and Humphrey, V. F. (2016), "Characterisation
	of sheet cavity noise of a hydrofoil using the Ffowcs Williams-Hawkings acoustic
	analogy". Computers \& Fluids, 130, 8-23
	\cite{Lidtke2016}
	\\ (doi:10.1016/j.compfluid.2016.02.014)
	\\ \url{https://eprints.soton.ac.uk/389848/}
	\\ \url{http://www.sciencedirect.com/science/article/pii/S004579301630038X}
%
\item Lidtke, A. K., Humphrey, V. F. and Turnock, S. R. (2015), "Feasibility study
	into a computational approach for marine propeller noise and cavitation modelling".
	Ocean Engineering, 120, July, pp. 152-159
	\cite{Lidtke2016a}
	\\ (doi:10.1016/j.oceaneng.2015.11.019).
	\\ \url{https://eprints.soton.ac.uk/385588/}
	\\ \url{http://www.sciencedirect.com/science/article/pii/S0029801815006320}
%
\item Marimon Giovannetti, L., Lidtke, A. K. and Taunton, D. J. (2014), "Investigation
	of tacking strategies using an America's cup 45 catamaran simulator", Procedia
	Engineering, 72, pp. 811-816
	\cite{Giovannetti2014}
	\\ (doi:10.1016/j.proeng.2014.06.137)
	\\ \url{https://eprints.soton.ac.uk/372719/}
	\\ \url{http://www.sciencedirect.com/science/article/pii/S1877705814006535}
%
\end{itemize}

%====
\subsection{Peer-reviewed conference articles}
%
\begin{itemize}
%
\item Lafeber, F-H, Bosschers, J., Lidtke, A. K., Lloyd, T. P. (2022),
	"Prediction of Underwater Radiated Noise from Propeller Cavitation During Concept Design",
	In 7th International Symposium of Marine Propulsors (SMP2022), 16-18 October, Wuxi, China
	\cite{lafeber_prediction_2022}
	\\ \url{https://www.researchgate.net/publication/365233056_Prediction_of_Underwater_Radiated_Noise_from_Propeller_Cavitation_During_Concept_Design}
%
\item Lloyd, T. P., Lidtke, A. K., Kerkvliet, M.,  Bosschers, J. (2022),
	"Broadband Trailing-Edge Noise Predictions using Incompressible Large Eddy Simulations",
	In 28th AIAA/CEAS Aeroacoustics 2022 Conference, Southampton, UK
	\cite{lloyd_broadband_2022}
	\\ \url{https://www.researchgate.net/publication/361282467_Broadband_Trailing-Edge_Noise_Predictions_using_Incompressible_Large_Eddy_Simulations}
%
\item Rijsbergen, M. X., Lidtke, A. K., Lajoinie, G. and Versluis M. (2020),
	"Sheet Cavitation Inception Mechanisms on a NACA 0015 Hydrofoil",
	In 33rd Symposium on Naval Hydrodynamics (SNH), Osaka, Japan, October 18 – 23, 2020
	\cite{Rijsbergen2020}
	\\ \url{https://www.marin.nl/publications/sheet-cavitation-inception-mechanisms-on-a-naca-0015-hydrofoil}
	\\ \url{https://www.researchgate.net/publication/333395191\_Acoustic\_modelling\_of\_a\_propeller\_subject\_to\_non-uniform\_inflow}
%
\item Lidtke, A. K., Lloyd, T. P. and Vaz, G. (2019),
	"Acoustic modelling of a propeller subject to non-uniform inflow",
	In Sixth International Symposium on Marine Propulsors (SMP), Rome, Italy, 26-30 May, 2019
	\cite{Lidtke2019a}
	\\ \url{https://www.marin.nl/acoustic-modelling-of-a-propeller-subject-to-non-uniform-in-flow}
	\\ \url{https://www.researchgate.net/publication/333395191\_Acoustic\_modelling\_of\_a\_propeller\_subject\_to\_non-uniform\_inflow}
%
\item Higgens, A. D., Lidtke, A. K., Joseph, P. F. and Turnock, S. R. (2018),
	"Investigation into the Tip Gap Flow and its Influence on Ducted Propeller Tip Gap Noise Using Acoustic Analogies",
	In 32nd Symposium on Naval Hydrodynamics (SNH), Hamburg, Germany, 5-10 August
	\cite{Higgens2018}
	\\ \url{https://www.researchgate.net/publication/333934662\_Investigation\_into\_the\_Tip\_Gap\_Flow\_and\_its\_Influence\_on\_Ducted\_Propeller\_Tip\_Gap\_Noise\_Using\_Acoustic\_Analogies}
%
\item Lidtke, A. K., Turnock, S. R. and Downes, J. (2018),
	"Characterising Influence of Transition to Turbulence on the Propulsive Performance of Underwater Gliders",
	of cavitation-induced pressure around the delft twist 11 hydrofoil",
	In 32nd Symposium on Naval Hydrodynamics (SNH), Hamburg, Germany, 5-10 August
	\cite{Lidtke2018}
	\\ \url{https://www.researchgate.net/publication/331281060_Characterizing_Influence_of_Transition_to_Turbulence_on_the_Propulsive_Performance_of_Underwater_Gliders}
%
\item Lidtke, A. K., Turnock, S. R. and Humphrey, V. F. (2016), "Multi-scale modelling
	of cavitation-induced pressure around the delft twist 11 hydrofoil",
	In 31st Symposium on Naval Hydrodynamics (SNH), Monterey, CA, USA, 11 - 16 Sep 2016
	\cite{Lidtke2016b}
	\\ \url{https://eprints.soton.ac.uk/401528/}
%
\item Lidtke, A. K., Humphrey, V. F. and Turnock, S. R. (2015), "Use of Acoustic
	Analogy for Marine Propeller Noise Characterisation", In Fourth International
	Symposium on Marine Propulsors (SMP), Autsin, TX, USA, 31 May - 4 June, 2015
	\cite{Lidtke2015a}
	\\ \url{https://eprints.soton.ac.uk/386511/}
%
\end{itemize}

%====
\subsection{Conference articles and workshop items}
%
\begin{itemize}
%
\item Kara, E., Lidtke, A. K., Rijpkema, D.,  D{\"u}z, B. , Kemal Kinaci, O. (2023)
	"Quantifying uncertainties in numerical predictions of dynamic cavitation"
	In 5th ECCOMAS Thematic Conference on Uncertainty Quantification in Computational Sciences and Engineering
	12—14 June, Athens, Greece
	\cite{kara_quantifying_2023}
	\\ \url{https://www.researchgate.net/publication/369322866_Quantifying_uncertainties_in_numerical_predictions_of_dynamic_cavitation}
%
\item  Lidtke, A. K., Rijpkema, D.,  D{\"u}z, B. (2023)
	"Combining deep reinforcement learning and computational fluid dynamics for efficient navigation in turbulent flows"
	In X International Conference on Computational Methods in Marine Engineering (MARINE)
	27-29 June, Madrid, Spain
	\cite{lidtke_combining_2023}
	\\ \url{https://www.researchgate.net/publication/369321603_Combining_deep_reinforcement_learning_and_computational_fluid_dynamics_for_efficient_navigation_in_turbulent_flows}
%
\item Scussel, A., Lidtke, A. K., D{\"u}z, B., Rijpkema, D. (2022)
	"Uncertainty quantification in the prediction of cavitation inception of the Duisburg Propeller Test Case"
	In 24th Numerical Towing Tank Symposium (NuTTS)
	October, Zagreb, Croatia
	\cite{scussel_uncertainty_2022}
	\\ \url{https://www.researchgate.net/publication/366085405_Uncertainty_quantification_in_the_prediction_of_cavitation_inception_of_the_Duisburg_Propeller_Test_Case}
%
\item Wielgosz, Ch., Golf, R., Lidtke, A. K., Vaz, G., el Moctar, O. (2019),
	"Numerical and experimental study on the Duisburg Propeller Test Case"
	In 22nd Numerical Towing Tank Symposium (NuTTS)
	29 Sep - 1 October, Tomar, Portuga
	\cite{Wielgosz2019}
	\\ \url{https://www.researchgate.net/publication/336564093_Numerical_and_experimental_study_on_the_Duisburg_Propeller_Test_Case}
%
\item Katsuno, E. T., Lidtke, A. K., D{\"u}z, B., Rijpkema, D., Vaz, Guilherme (2019)
	"Parameter Uncertainty Quantification applied to the Duisburg Propeller Test Case"
	In 22nd Numerical Towing Tank Symposium (NuTTS)
	29 Sep - 1 October, Tomar, Portuga
	\cite{Katsuno2019}
	\\ \url{https://www.researchgate.net/publication/336564327_Parameter_Uncertainty_Quantification_applied_to_the_Duisburg_Propeller_Test_Case}
%
\item Wang, T., Lidtke, A. K., Giorgio-Serchi, F. and Weymouth, G. D. (2019),
	"Manoeuvring of an aquatic soft robot using thrust-vectoring",
	In 2nd IEEE International Conference on Soft Robotics (RoboSoft),
	Seoul, South Korea, 14-18 April, pp. 186-191
	\cite{Wang2019}
	\\ \url{https://ieeexplore.ieee.org/abstract/document/8722732}
	\\ \url{https://www.researchgate.net/publication/333427988\_Manoeuvring\_of\_an\_aquatic\_soft\_robot\_using\_thrust-vectoring}
%
\item Lidtke, A. K., Giorgio-Serchi, F., Lisle, M. and Weymouth, G. D. (2018),
	"A low-cost experimental rig for multi-DOF unsteady thrust measurements of aquatic bioinspired soft robots",
	In IEEE-RAS International Conference on Soft Robotics, Livorno, Italy, 24-28 April
	\cite{Lidtke2018a}
	\\ \url{https://ieeexplore.ieee.org/abstract/document/8405377}
	\\ \url{https://eprints.soton.ac.uk/418281/}
%
\item Lidtke, A. K., Turnock, S. R. and Downes, J. (2017),
	"Simulating turbulent transition using Large Eddy Simulation with application to underwater vehicle hydrodynamic modelling",
	In 20th Numerical Towing Tank Symposium (NuTTS), Wageningen, Netherlands, 1-3 October 2017
	\cite{Lidtke2017a}
	\\ \url{https://eprints.soton.ac.uk/428887/}
	\\ \url{https://www.researchgate.net/publication/331770591\_Simulating\_turbulent\_transition\_using\_Large\_Eddy\_Simulation\_with\_application\_to\_underwater\_vehicle\_hydrodynamic\_modelling}
%
\item Lidtke, A. K., Lewis, S., Harvey, T., Turnock, S. R. and Downes, J. (2017),
	"An experimental study into the effect of transitional flow on the performance
	of underwater glider wings", In Oceans '17 MTS/IEEE Aberdeen, United Kingdom.
	19 - 22 Jun 2017. 10 pp.
	\cite{Lidtke2017b}
	\\ \url{https://eprints.soton.ac.uk/411106/}
%
\item Lidtke, A. K., Turnock, S. R. and Downes, J. (2016), "Assessment of underwater
	glider performance through viscous computational fluid dynamics", In Autonomous
	Underwater Vehicles (AUV 2016), Tokyo, Japan, 6 - 9 Nov 2016, IEEE Oceanic Engineering Society (IEEE-OES)
	\cite{Lidtke2016d}
	\\ \url{https://eprints.soton.ac.uk/400409/}
	\\ \url{http://ieeexplore.ieee.org/document/7778698/}
%
\item Lemaire, S., Lidtke, A. K., Vaz, G. and Turnock, S. R. (2016), "Modelling
	natural transition on hydrofoils for application in underwater gliders",
	In 19th Numerical Towing Tank Symposium (NUTTS'16), St Pierre d'Oleron, Fance, 3 - 4 Oct 2016
	\cite{Lemaire2016}
	\\ \url{https://eprints.soton.ac.uk/401533/}
	\\ \url{http://www.marin.nl/web/News/News-items/Modelling-natural-transition-on-hydrofoils-for-application-in-underwater-gliders-1.htm}
%
\item Lidtke, A. K., Lakshmynarayanana, A., Camilleri, J., Banks, J., Phillips, A. B.,
	Turnock, S. R. and Badoe, C. (2015), "RANS computations of flow around a bulk
	carrier with energy saving device", In Tokyo 2015: A Workshop on CFD in Ship
	Hydrodynamics, Tokyo, Japan, 2 - 4 Dec 2014
	\cite{Lidtke2015b}
	\\ \url{https://eprints.soton.ac.uk/388503/}
%
\item Lloyd, T. P., Lidtke, A. K., Rijpkema, D. R., van Wijngaarden, H. C. J.,
	Turnock, S. R. and Humphrey, V. F., "Using the FW-H equation for hydroacoustics
	of propellers", In 18th Numerical Towing Tank Symposium (NuTTS), Cortona, Italy, 28-30 September 2015
	\cite{Lloyd2015a}
	\\ \url{http://www.marin.nl/web/Publications/Publication-items/Using-the-FWH-equation-for-hydroacoustics-of-propellers.htm}
%
\item Badoe, C., Winden, B., Lidtke, A. K., Phillips, A. B., Hudson, D. A. and
	Turnock, S. R. (2014), "Comparison of various approaches to numerical simulation
	of ship resistance and propulsion", In SIMMAN 2014, Lyngby, Denmark, 8 - 10 Dec 2014
	\cite{Badoe2014}
	\\ \url{https://eprints.soton.ac.uk/369875/}
%
\item Lidtke, A. K., Turnock, S. R. and Humphrey, V. F., "Outlook on Marine Propeller
	Noise and Cavitation Modelling", A. Yucel Odabasi (AYO) Colloquium Series, Istanbul,
	Turkey, 6-7 November 2014
	\cite{Lidtke2014a}
	\\ \url{https://eprints.soton.ac.uk/385639/}
%
\item Lidtke, A. K., Turnock, S. R. and Humphrey, V. F., "The influence of turbulence
	modelling techniques on the predicted cavitation behaviour on a NACA0009 foil",
	In 17th Numerical Towing Tank Symposium (NuTTS), Gothenburg, Sweden, 28-30 Sep 2014
	\cite{Lidtke2014}
	\\ \url{https://eprints.soton.ac.uk/386513/}
%
\item Lidtke, A. K., Marimon Giovanetti, L., Breschan, L. M., Sampson, A., Vitti, M.
	and Taunton, D. J. (2013), "Development of an America's Cup 45 tacking simulator",
	In Third International Conference on Innovation for High Performance Sailing
	Yachts INNOV’SAIL, Lorient, Fance, 26 - 28 Jun 2013
	\cite{Lidtke2013}
	\\ \url{https://eprints.soton.ac.uk/386514/}
%
\end{itemize}

%====
\subsection{Periodical articles}
%
\begin{itemize}
\item Lidtke, A. K., Turnock, S. R. and Humphrey, V. F. (2016), "Saving ocean sound
	scapes", The Naval Architect, pp. 36-38
	\cite{Lidtke2016c}
	\\ \url{https://eprints.soton.ac.uk/386510/}
%
\end{itemize}

%====
\subsection{Books and public reports}
%
\begin{itemize}
\item Lidtke, A. K. (2017), "Predicting radiated noise of marine propellers using acoustic analogies and hybrid Eulerian-Lagrangian cavitation models,
	Doctoral thesis, University of Southampton, United Kingdom
	\cite{Lidtke2017d}
	\\ \url{https://eprints.soton.ac.uk/413579/}
%
\item Lidtke, A. K. (2017), "OpenFOAM programming tutorials for beginners," [Online]
	\cite{Lidtke2017c}
	\\ \url{https://github.com/UnnamedMoose/BasicOpenFOAMProgrammingTutorials}
%
\item Shenoi, R. A., Bowker, J. A., Dzielendziak, A. S., Lidtke, A. K., Zhu, G., Cheng, F.,
	Argyos, D., Fang, I., Gonzalez, J., Johnson, S., Ross, K., Kennedy, I., O'Dell, M.
	and Westgarth, R. (2015), "Global marine technology trends 2030", Southampton, GB,
	University of Southampton, 186pp
	\cite{Shenoi2015}
	\\ \url{https://eprints.soton.ac.uk/388628/}
%
\end{itemize}

\newpage

%%% ABSTRACTS %%%
%=====================
\section{Abstracts}

%---------------------
\subsection{Quantifying uncertainties in numerical predictions of dynamic cavitation \cite{kara_quantifying_2023}}

Cavitation on marine propellers is an important issue due to its negative effects on many aspects of their operation. Therefore, accurate prediction of cavitation is important to ensure better propeller design. Estimating the cavitation behavior numerically is a~difficult task due to the high computational cost of simulations and various numerical uncertainties. This study is carried out in order to estimate the parameter and discretization uncertainties and combine them into a~single value using an example 2D foil as the test case. Angle of attack and the cavitation number are selected as input parameters and their influence on force coefficients and sheet cavity properties is studied. Sobol indices are also obtained to measure the relative importance of the input and discretization uncertainties. It is seen that uncertainty in the angle of attack has a~much greater influence on the force coefficients than the uncertainty in the cavitation number or grid discretization uncertainty. On the other hand, the cavitation number uncertainty is dominant over the grid and angle of attack uncertainties for the length and volume of the cavity sheet. According to the results obtained, applying only the grid refinement studies is not sufficient for the estimation of the numerical uncertainties for this kind of CFD problems. It is proposed that assessing both parameter and discretization uncertainties for the presented and other similar applications with epistemic uncertainties should be applied more often.

%---------------------
\subsection{Combining deep reinforcement learning and computational fluid dynamics for efficient navigation in turbulent flows \cite{lidtke_combining_2023}}

Autonomous underwater vehicles (AUVs) face significant challenges when navigating in turbulent environments, particularly when carrying out tasks such as inspecting offshore structures that generate large turbulent wakes. These environments increase the risk of collision and damage, and decrease the success rate of recorded video frames, but carrying out such inspections with AUV offers large potential cost savings and reduced risk to human operators. Reinforcement learning (RL) combined with computational fluid dynamics (CFD) can help develop control strategies suitable for handling such complex navigation problems. The objective of this study is to assess the feasibility of such approach. To this end, two versions of the soft actor-critic algorithm are tested: one relying on the estimated vehicle position and velocity and the other augmented with pressure surface measurements obtained from fitting the vehicle with simulated pressure transducers. Both RL agents successfully navigate in turbulent flows, but the agent provided with force estimates deduced from the surface pressure has significantly improved performance. This improvement is seen in the quality of individual episodes as well as in the training robustness and speed. Therefore, this study demonstrates the potential of using RL agents to assimilate additional information for developing robust control strategies and shows the usefulness of training RL agents in high-fidelity environments, such as CFD simulations.

%---------------------
\subsection{Predicting cavitating propeller noise in off-design conditions using scale-resolving CFD simulations \cite{lidtke_predicting_2022}}

There is increasing awareness about the harmful impact of underwater radiated noise of shipping on the marine environment, with propeller cavitation being a major contributor thereof. In order to allow low-noise propeller design, reliable and validated numerical tools are necessary. The combined use of viscous computational fluid dynamics (CFD) and Ffowcs Williams-Hawkings acoustic analogy has long been suggested as a potential frontrunner that could address this need. However, few studies presented in the open literature have shown detailed validation focused on farfield radiated noise of propellers in cavitating conditions. Present work aims to address this by applying the methodology to two thrusters operating in off-design conditions and tested at model scale. Flow is computed using scale-resolving CFD simulations and a mass-transfer cavitation model. This allows for part of the turbulence spectrum and cavitation dynamics to be resolved. It is shown that peak sound pressure levels, corresponding to the low-frequency underwater radiated noise source, may be predicted to within 5 dB of experimental results. In addition, key features of the noise spectra, such as centre frequency of the peak broadband noise level and decay slope, are also well represented in the computations. The results are supplemented by analysis of the numerical signal-to-noise ratio.

%---------------------
\subsection{Prediction of Underwater Radiated Noise from Propeller Cavitation During Concept Design \cite{lafeber_prediction_2022}}

There is growing concern about the impact of underwater radiated noise (URN) on marine life. One of the main sources of URN of ships is propeller cavitation. Semi-empirical computational models to predict back (suction) side cavitation at the design point of open propellers have been published, but there is a lack of models that predict the URN of open propellers in off-design conditions and the URN of ducted propellers, such as thrusters. The European Union NAVAIS 1 project considered two ship types that spend large proportions of time operating at off-design conditions-a road ferry and an aquaculture workboat-and are therefore likely to experience several different forms of propeller cavitation. The present paper discusses new semi-empirical models to be used together with a boundary element method for predicting noise from these forms of cavitation. The new (medium-fidelity) models were tuned using data from a large series of model-scale noise measurements, supplemented by high-fidelity scale-resolving computational fluid dynamics simulations combined with the Ffowcs Williams-Hawkings acoustic analogy. The medium-fidelity models were used to predict the URN from a large series of propellers for a wide range of operating conditions, with the results used in a regression analysis to develop a low-fidelity tool for estimating propeller URN of road ferries and workboats during the concept design phase.

%---------------------
\subsection{Broadband Trailing-Edge Noise Predictions using Incompressible Large Eddy Simulations \cite{lloyd_broadband_2022}}

Incompressible large eddy simulations of the flow over an airfoil at high Reynolds number and low Mach number have been performed, including simulation of bypass transition to turbulence, and trailing-edge noise predictions. Extensive data comparison and validation has been carried out, using benchmark data and additional results from literature. Three semi-analytical noise models - Curle's acoustic analogy, the Ffowcs Williams-Hall (FW-Hall) model, and Amiet's model - were applied, with the sensitivity of the results to the hydrodynamic input data investigated. Overall, a good agreement was found between experimental and numerical results in terms of hydrodynamics. The main discrepancies were an overprediction of the Reynolds stresses close to the wall, and the high-frequency part of the turbulent wall pressure spectrum, both at the trailing edge. It is hypothesised that this is caused by the underresolved transition of the boundary layer. In terms of the acoustic predictions, the FW-Hall and Amiet models perform better than Curle’s analogy, with Amiet giving the best agreement with measurements below 2 kHz, and FW-Hall being superior above this frequency, with the latter model identified as being less sensitive to the aforementioned errors in the hydrodynamic results.

%---------------------
\subsection{Uncertainty quantification in the prediction of cavitation inception of the Duisburg Propeller Test Case \cite{scussel_uncertainty_2022}}

Accurate quantitative prediction of marine propeller performance and cavitaiton inception criteria has important implications for the eventual ship operator. However, there exist a wide range of uncertain parameters that need to be input to the CFD simulations and whose choice will affect the predictions made. As stated by NASA CFD 2030 outlook, the management of errors and uncertainties due to the lack of knowledge in parameters of a given fluidic problem is an important trend for studies involving computer simulations. Aiming to mitigate gaps in studies regarding the cavitation inception of marine propellers and to deepen the discussion on the subject, the present study seeks to quantify the uncertainties in the input data of a CFD simulation of the Duisburg propeller P1570. An existing uncertainty quantification (UQ) Python framework of Katsuno et al. (2021) capable of calculating the confidence interval (CI) and the Sobol indices (S) of each input variable is further developed and used to deduce the effect of physical input parameters on the simulation results.

%---------------------
\subsection{Scale-Resolving Simulations of a Circular Cylinder Subjected to Low Mach Number Turbulent Inflow \cite{Lidtke2021b}}

Inflow turbulence is relevant for many engineering applications relating to noise generation, including aircraft wings, landing gears, and non-cavitating marine propellers. While modelling of this phenomenon is well-established for higher Mach number aerospace problems, lower Mach number applications, which include marine propellers, still lack validated numerical tools. For this purpose, simplified cases for which extensive measurement data are available can be used. This paper investigates the effect of inflow turbulence on a circular cylinder at a Reynolds number of 14,700, a Mach number of 0.029, and with inflow turbulence intensities ranging between 0\% and 22\%. In the present work focus is put on the hydrodynamics aspect, with the aim of addressing radiated noise in a later study. The flow is simulated using the partially averaged Navier Stokes equations, with turbulence inserted using a synthetic inflow turbulence generator. Results show that the proposed method can successfully replicate nearfield pressure variations and relevant flow features in the wake of the body. In agreement with the literature, increasing inflow turbulence intensity adds broadband frequency content to all the presented fluctuating flow quantities. In addition, the applied variations in inflow turbulence intensity result in a major shift in flow dynamics around a turbulence intensity of 15\%, when the dominant effect of von Kármán vortices on the dominant flow dynamics becomes superseded by freestream turbulence.

%---------------------
\subsection{Parameter uncertainty, Discretization uncertainty, Uncertainty quantification analysis, Sobol indices, Transitional flow, CFD \cite{Katsuno2021}}

For computational fluid dynamic (CFD) simulations at intermediate Reynolds numbers, laminar-to-turbulent transition plays an important role in the flow physics. While this phenomenon is difficult to simulate, current state-of-the-art transition models provide viable means to do so without resorting to expensive scale-resolving simulations. Previous studies of these methods show that, as in reality, the results are sensitive to the inlet turbulence characteristics and their decay upstream of the studied body. The lack of understanding of the exact effect of these boundary condition values may contribute, together with the discretization uncertainties, to an increase of uncertainty in the output quantities of interest, such as force or torque values. Using the $\gamma$-Re-$\theta$ transition model, this work estimates the parameter and discretization uncertainties and combines them into a single value. Three input uncertainties related to turbulence are chosen: turbulence intensity, eddy viscosity, and a parameter used to define a turbulence decay-free zone upstream. Sobol indices are also obtained in order to quantify the relative importance of input and discretization uncertainties. Two cases are tested: a flat plate, focusing on the local skin friction; and a practical case of the Duisburg Propeller Test Case (DPTC), where surface-integral quantities of thrust and torque coefficients are assessed. Both cases show that the turbulence intensity and eddy viscosity have significant values of total-effect Sobol indices, showing a considerable contribution of these factors to the output variation of quantity of interests, making them key factors affecting the sensitivity of the final result. Results suggest that performing grid refinement studies alone is not sufficient to estimate the numerical uncertainties for this category of problems. It is therefore recommended to assess parameter and discretization uncertainties together for the presented and other similar applications with epistemic uncertainties related to turbulent kinetic energy.

%---------------------
\subsection{End-to-end efficiency quantification of an autonomous underwater vehicle propulsion system \cite{Lidtke2021a}}

Increasing demand for versatile and long-endurance autonomous underwater vehicles puts significant design pressure on all aspects of AUV design and operation, including that of the propulsive system. The present study discusses testing of a thruster unit and several propellers developed to propel a hybrid glider/flight-style underwater vehicle. Due to the AUV being required to operate at largely different speeds and thrust levels between the two configurations, the propulsive subsystem needs to be capable of remaining efficient and effective across a wide range of operating conditions. Thus, the current results focus on quantifying all of the factors affecting the drive train, ranging from open-water performance of the propeller up to electro-mechanical efficiency of the magnetic coupling and geared electric motor. It is shown that, depending on the required operating point, total efficiency of the vehicle is primarily affected by non-linear low Reynolds number effects, sudden drop of gearbox efficiency at low revolutions and applied torques, as well as blade deformation, aside of the baseline propeller efficiency.

%---------------------
\subsection{Sheet Cavitation Inception Mechanisms on a NACA 0015 Hydrofoil \cite{Rijsbergen2020}}

Experimentally measured trajectories of a series of free-stream nuclei and their growth while passing an isolated roughness element on a hydrofoil are compared with Lagrangian tracking computations on bubbles with the same initial size and position. The comparison shows a good agreement between the measurements and simulations, indicating that the proposed numerical tools may be used in order to determine cavitation inception criteria on hydrofoils and complement the interpretation of experiments. It is shown that the nuclei present in the experiments are not likely to have been fully gaseous; the combination of their trajectory and radius history indicate that they were solid particles with a gas layer around them.
Two series of experiments also investigated the similarities and differences between cavitation inception induced by free-stream nuclei passing over a roughness element and pits placed around the minimum pressure point. The protruding roughness element causes bubble expansion due to hydrodynamic pressure reduction and traps gas and water vapour in its wake. The (hydrophilic) pits trap gas and water vapour from a passing bubble. The cavity retracts in the pit with increasing ambient pressure until it becomes invisible. The bubble in the pit expands again with decreasing ambient pressure to a vaporous cavity.

%---------------------
\subsection{Investigation into the Tip-Gap Flow and Its Influence on Ducted Propeller Tip-Gap Noise Using Acoustic Analogies \cite{Higgens2020}}

	Ducted propellers are commonly used on naval vessels such as submarines and increasingly more in omnipresent autonomous underwater vehicles. Understanding the noise signature of these vessels is of critical importance from both detection and concealment perspective, but the detailed physics of how the tip of the propeller blade interacts with the boundary layer on the duct is not sufficiently well understood. The present study introduces a series of simulations in which tip vortices formed over finite-span lifting surfaces are investigated with the aim of better understanding interactions of tip-gap flow and its interaction with incident boundary layers on radiated noise. Particular attention is devoted to characterizing the nature of the incipient complex flow features and their effect on the force coefficients, surface pressure fluctuations. These reveal that variations of force coefficients with tip-gap height may be used as an early indicator of tip vortex being suppressed as the gap is reduced. Use of the lambda-2 criterion identifies the formation of strong coherent vortical structures in the tip-gap region. It is suggested that the interaction of these structures with the tip edges is likely to be an important source of noise, with a similar generation mechanism to trailing edge noise. Preliminary analysis of the acoustic signatures computed using the Ffowcs Williams–Hawkings acoustic analogy indicates high levels of directivity, with most of the noise being generated by the foil rather than the duct. The presence of substantial amounts of vorticity in the flow also suggests that accounting for the nonlinear quadrupole noise sources within the fluid might be necessary to fully describe the noise pattern developed.

%---------------------
\subsection{Parameter Uncertainty Quantification applied to the Duisburg Propeller Test Case \cite{Katsuno2019}}

	At model-scale conditions (diameter-based Reynolds number below 10 million), laminar-to-turbulent transition plays an important role on the performance of a propeller. While at these Reynolds numbers turbulent transition and its effects on the flow are difficult to simulate, current state-of-the-art transition models (Menter et al. (2004)) provide viable means to do so. Previous studies on these new methods (Ec¸a et al. (2016), Baltazar et al. (2018), Lopes et al. (2018)) show that, as in reality, the results are sensitive to the inlet turbulence intensity and its decay upstream of the propeller plane. In the aforementioned studies, numerical uncertainties were reported to have been small. Therefore, it is expected that variability of the user-specified inflow conditions will have a dominant effect on the solution. To better understand the associated uncertainty of model-scale open-water propeller performance predictions, Uncertainty Quantification (UQ) methods are employed in this work. Given a prescribed range of input uncertainties reflecting typical ranges found in test facilities and RANS simulation setups, a laminar-turbulent transition model is used in CFD simulations of the Duisburg Propeller Test Case. This paper then aims to use the results in order to quantify the parameter uncertainty and obtain the output-variable’s Cumulative Distribution Function (CDF), confidence interval and the Sobol indices of each input variable. Together, these quantities are used to suggest suitable ways of achieving accurate and repeatable predictions of propeller performance at intermediate Reynolds numbers.

%---------------------
\subsection{Numerical and experimental study on the Duisburg Propeller Test Case \cite{Wielgosz2019}}

	Understanding of cavitation behaviour on marine propellers is of critical importance to ship designers as it dictates several of the operating limits of the propulsor due to onset of increased erosion risk or unacceptable levels of noise and vibration. Consequently, this topic continues to inspire experimental studies aimed at providing a more in-depth understanding of the physical phenomena involved, but also to provide means of validation for numerical models. Unfortunately, many of these studies do not cover the complete spectrum of types of cavitating flows seen in practice on modern ship propellers and do not present uncertainties of the experimental data. Present work aims to address these issues by reporting on a new series of quantitative and qualitative model-scale propeller tests on the Duisburg Propeller Test Case (DPTC). The experimental data is then used to validate Computational Fluid Dynamics (CFD) predictions of open water performance and cavitation patterns and a verification and validation study is carried out.

%---------------------
\subsection{Acoustic modelling of a propeller subject to non-uniform inflow \cite{Lidtke2019a}}

	Understanding radiated noise from marine propellers is of interest from the point of view of mitigating sound pollution and avoiding detection. At the design stage this can be achieved using acoustic tools such as the Ffowcs Williams-Hawkings (FW-H) acoustic analogy. However, these have not yet been widely adopted and thoroughly tested for realistic configurations in the marine context. The current work presents a systematic study of applying the porous FW-H method, coupled with a viscous RANS solver, to the IN-SEAN E779A propeller in a wake field, in both cavitating and non-cavitating conditions. The aim is to understand sensitivity of the analogy to the definition of the porous data surfaces and key simulation parameters, such as time step and grid resolution. These will in turn provide base-line guidelines to be used in subsequent work devoted to fully appended ship predictions. The results indicate that particular care must be taken to define the porous data surfaces in such a way so as to minimise the amount of upstream vorticity penetrating them while ensuring the effect of noise-generating flow features is aptly captured.

%---------------------
\subsection{Manoeuvring of an aquatic soft robot using thrust-vectoring \cite{Wang2019}}

	Capability of a pulsed-jetting, aquatic soft robot to perform turning manoeuvres by means of a steerable nozzle is investigated experimentally for the first time. Actuation of this robot is based on the periodic conversion of slowly-charged elastic potential energy into fluid kinetic energy, giving rise to a cyclic pulsed-jet resembling the one observed in cephalopods. A steerable nozzle enables the fluid jet to be deflected away from the vehicle axis, thus providing the robot with the unique ability to manoeuvre using thrust-vectoring. This actuation scheme is shown to offer a high degree of control authority when starting from rest, yielding turning radii of the order of half of the body length of the vehicle. The most significant factor affecting efficiency of the turn has been identified to be the fluid momentum losses in the deflected nozzle. This leads, given the current nozzle design, to a distinct optimum nozzle angle of 35 degrees.

%---------------------
\subsection{Characterizing Influence of Transition to Turbulence on the Propulsive Performance of Underwater Gliders \cite{Lidtke2019}}

	Two models of underwater gliders were tested in a wind tunnel, one corresponding to a legacy shape commonly used in contemporary vehicles, the other being a scaled-down version of a new design. Performance of the two vehicles was characterised over a range of speeds and angles of attack. Particular attention was paid to the effect of sharp features along the hulls of the two vehicles and how they affect the observed flow regime. It has been shown that the new design, which employs a bow shaped to encourage natural laminar flow, benefits from a 10\% reduction of parasitic drag and 13\% increase in L/D when the hull surface is smooth. The legacy glider, made up of a faired bow and a cylindrical hull, suffers from laminar separation and up to 100\% increase in induced drag if the flow over its bow is prevented from transitioning to a turbulent state before encountering adverse pressure gradient at lower Reynolds numbers. This results in lowering of attainable speed at shallow glide path angles while the associated parasitic drag reduction is demonstrated to increase the maximum velocity of the glider when moving at glide slopes above approximately 30.

%---------------------
\subsection{A soft aquatic actuator for unsteady peak power amplification \cite{GiorgioSerchi2018}}

	A soft hydraulic actuator is presented which uses elastic energy storage for the purpose of pulsed-jet propulsion of soft unmanned underwater vehicles. The actuator consists of a flexible membrane which can be inflated using a micro-pump and whose elastic potential energy may be released on demand using a controllable valve, in a manner inspired by the swimming of squids and octopuses. It is shown that for equivalent initial elastic energy, the drop in peak thrust is linearly proportional to the decrease in nozzle cross section. Peak hydraulic power amplification of the soft actuator of approximately 75\% is achieved with respect to that of the driving pump, confirming that passive elasticity can be exploited in aquatic propulsion to replicate the explosive motion skills of agile sea-dwelling creatures.

%---------------------
\subsection{Investigation into the Tip Gap Flow and its Influence on Ducted Propeller Tip Gap Noise Using Acoustic Analogies \cite{Higgens2018}}

	Ducted propellers are commonly used on naval vessels such as submarines as well as increasingly more omnipresent autonomous underwater vehicles (AUVs). Understanding the noise signature of these vessels is of critical importance both from detection and concealment perspective, but the detailed physics of how the tip of the propeller blade interacts with the boundary layer on the duct are not sufficiently well understood. Presesnt study introduces a series of simulations in which tip vortices formed over finite-span lifiting surfaces are investigated with the aim of better understanding interactions of tip-gap flow and its interaction with incident boundary layers on radiated noise. Particular attention is devoted to characterising the nature of the incipient complex flow features and their effect on the force coefficients, surface pressure fluctuations. These reveal that variations of force coefficients with tip-gap height may be used as an early indicator of tip vortex being suppressed as the gap is reduced. Analysis of the acoustic signatures computed using Ffowcs Williams-Hawkings acoustic analogy indicates highl levels of directivity with majority of the noise being generated by the foil rather than the duct. Presence of substantial amounts of vorticity in the flow also suggests that accounting for the non-linear quadrupole noise sources within the fluid might be necessary to fully describe the noise pattern developed.

%---------------------
\subsection{A low-cost experimental rig for multi-DOF unsteady thrust measurements of aquatic bioinspired soft robots \cite{Lidtke2018a}}

	The design, calibration and testing of an experimental rig for measuring 2-DOFs unsteady loads over aquatic robots is discussed. The presented apparatus is specifically devised for thrust characterization of a squid-inspired soft unmanned underwater vehicle, but its modular design lends itself to more general bioinspired propulsion systems and the inclusion of additional degrees of freedom. A purposely designed protocol is introduced for combining calibration and error compensation upon which force and moment measurements can be performed with a mean error of 0.8\% in steady linear loading and 1.7\% in unsteady linear loading, and mean errors of 10.2\% and 9.4\% respectively for the case of steady and dynamic moments at a sampling rate of the order of 10 Hz. The ease of operation, the very limited cost of manufacturing and the degree of accuracy make this an invaluable tool for fast prototyping and low-budget projects broadly applicable in the soft robotics community.

%---------------------
\subsection{Simulating turbulent transition using Large Eddy Simulation with application to underwater vehicle hydrodynamic modelling \cite{Lidtke2017a}}

	Large Eddy Simulation (LES) has been widely used by the aerospace community in order to model laminar separation bubbles and other low Reynolds number phenomena. In maritime-related applications this family of turbulence modelling techniques has typically been used to model unsteady cavitation. Present work aims to apply it to develop first-hand experience with modelling laminar separation bubbles using LES in OpenFOAM, specifically looking at the effects of the choice of the subgrid model. The investigation is carried out on the SD7003 2D foil section, for which PIV flow field measurements, as well as reference CFD results, are available. Four different popular LES models are tested: Smagorinsky, dynamic $k$-equation, wall-adaptive (WALE), and implicit (ILES). The ultimate goal of this work is to apply the established methodology to model flows on underwater vehicle appendages, as well as propellers, which have been reported to experience noticeable amounts of laminar flow when operating at model-scale Reynolds numbers. These flows experience complex, unsteady hydrodynamic phenomena, such as tip and root vortices, laminar separation bubbles, and are affected by onset turbulence. Thus, studying them with LES could lead to improved predictions compared to the previous work by the authors which relied on using RANS transition models to simulate the flow past underwater vehicle geometries.

%---------------------
\subsection{Hydrodynamic design of underwater gliders using k-k$_L$-$\omega$ RANS transition model \cite{Lidtke2017}}

	Hydrodynamic design of an underwater glider is an act of balancing the requirement for a streamlined, hydrodynamically effective shape and the consideration of the practical aspects of the intended operational envelope of the vehicle, such as its ability to deploy a wide range of sensors across the water column. Key challenges in arriving at a successful glider design are discussed and put them in the context of existing autonomous underwater vehicles (AUV) of this type. The design cycle of a new vehicle shape is then described. The discussed AUV will operate both as an buoyancy-propelled glider and a flight-style, propellerdriven submersible, utilising its large size to deliver substantial scientific payloads to remote locations to perform environmental monitoring, seabed survey, and exploration for sub-sea oil, gas and material deposits. Emphasis is put on using computational fluid dynamic (CFD) methods capable of predicting laminar-turbulent transition of the flow in order to estimate the performance of candidate designs and thus inform and guide the evolution of the vehicle. A range of considered shapes are therefore described and their hydrodynamic characteristics predicted using CFD are summarised. A final shape for the new glider is then proposed. This is then subject to an in-depth flow-field analysis which points out how natural laminar flow may be used as a means of drag reduction without compromising the practical aspects of the design, such as its ability to carry sufficient payload. Finally, the obtained data are used to project the expected glide paths, as well as give preliminary estimates of its range. These show the benefits of minimising the vehicle drag, as well as highlight the possible trade offs between maximising speed and endurance of the AUV.

%---------------------
\subsection{Predicting radiated noise of marine propellers using acoustic analogies and hybrid Eulerian-Lagrangian cavitation models \cite{Lidtke2017d}}

	Anthropogenic noise from a variety of merchant ships has been reported to be a major factor adversely aecting marine organisms. Consequently, scientists and regulators have become more vocal about encouraging, and possibly enforcing, quieter ships in the future. For this to be feasible from an engineering standpoint, a range of numerical methods must be made available to allow acoustic performance of vessels to be evaluated at the design stage. Cavitation is a major contributor to the hydroacoustic signature of a merchant vessel. The reason for this is the relatively high drop of pressure induced by the propeller, which in turn promotes the growth of vapour bubbles and cavities, oscillation and collapse of which act as strong acoustic sources. The entire process is made more dynamic by the non-uniform wake of the ship, propeller rotation, as well as the fact that vessels travel in a seaway. Because of its complexity, the problem of marine propeller noise is thus not widely studied numerically, which translates to the lack of tools readily available to designers willing to reduce the noise generated by ships. A set of numerical utilities are proposed which could be employed at the late design stage of a merchant ship in order to allow the designer to estimate the radiated noise and make informed decisions on how to improve the design. The methodology involves solving the turbulent flow over the propeller using Detached Eddy Simulation (DES) and modelling cavitation using a mass-transfer model. The porous Ffowcs Williams-Hawkings acoustic analogy is used to infer far-eld radiated noise caused by the blade rotation, pulsating cavitation, as well as non-linear noise sources in the propeller slip-stream. The cavitation model is also extended to incorporate Lagrangian bubbles dispersed downstream of the large cavities modelled using the baseline Schnerr-Sauer model via the volume fraction equation approach. This allows the broadband nature of cavitation noise to be captured. The methods are applied to a NACA 66 and the Delft Twist 11 hydrofoil test cases. Although there are limited validation data allowing all of the methods to be validated simultaneously, relatively good agreement is seen at intermediate validation stages. These include comparing the non-cavitating noise of the Insean E779a propeller to reference data, conducting acoustic predictions for idealised acoustic sources, as well as comparing cavitation patterns, cavity cloud shedding frequencies, and induced pressures to experimental data for hydrofoils and propellers. It is concluded that the presented methodology may be used to predict low-frequency noise due to cavitation in a relatively robust manner, although the method is yet to be tested and validated on more complex geometries. The hybrid Eulerian-Lagrangian model is still at an early stage and a range of areas for improvement have been identified, such as implementation of more realistic cavity break-up models as well as better coupling between the fluid and bubble solvers. Nonetheless, the method is demonstrated to be a promising tool at tackling the broadband cavitation noise components as it can capture the contribution of the mass of small, oscillating bubbles on the radiated pressure which would otherwise be unaccounted for in the baseline Eulerian framework.

%---------------------
\subsection{An experimental study into the effect of transitional flow on the performance of underwater glider wings \cite{Lidtke2017b}}

	The experimental procedures developed to measure small forces generated by underwater vehicle appendages in a wind tunnel are described. These involve using precision load cell sensors arranged into a bespoke dynamometer, that has been integrated into a stand-alone wind tunnel rig. The set up is used to quantify performance of a scaled underwater glider hydrofoil, with particular care being paid to the effects of laminar and transitional flow features. Present data are benchmarked against similar measurements and simulations and it is shown that satisfactory results may be obtained with the current approach despite its simplicity and low cost. The apparatus is then used to show that the presence of laminar flow significantly affects performance of the considered foil at Reynolds numbers corresponding to a typical regime in which underwater gliders operate.

%---------------------
\subsection{Assessment of underwater glider performance through viscous computational fluid dynamics \cite{Lidtke2016d}}

	The process of designing an apt hydrodynamic shape for a new underwater glider is discussed. Intermediate stages include selecting a suitable axi-symmetric hull shape, adding hydrofoils and appendages, and evaluating the performance of the final design. All of the hydrodynamic characteristics are obtained using computational fluid dynamics using the k-k$_L$-$\omega$ transition model. It is shown that drag reduction of the main glider hull is of crucial importance to the ultimate performance. Suggested steps for achieving this are the encouragement of natural laminar flow, integration of sensors into the streamlined hull shape, and sound operational practice.

%---------------------
\subsection{Modelling Natural Transition on Hydrofoils for Application in Underwater Gliders \cite{Lemaire2016}}

	Underwater gliders are a class of autonomous underwater vehicles (AUV) used for long-endurance missions. They employ a buoyancy engine in order to induce vertical motion through the water column and by adopting an appropriate trim angle they propel themselves forward using hydrofoils. The velocities these vessels reach are typically less than 0.5 m/s. Efficiency of their hydrofoils is of course a key factor determining the overall system performance. Therefore, better understanding the nature of transition occurring on these foils is of significant importance for the design of next generation underwater gliders. Standard turbulence models are known to predict transition onset too early in terms of Reynolds number, mainly because they were first developed with applications to fully turbulent flows in mind (Rosetti et al., 2016). For this reason a range of models designed to predict transition have been introduced in the literature (Van Ingen, 2008). The aim of this paper is to assess the usefulness of the Local Correlation Transition Model (LCTM, also known as gamma-Re-theta, Langtry and Menter, 2009), implemented in the finite volume solver ReFRESCO (Rosetti et al., 2016), for application to engineering problems involving laminar and transitional Reynolds number regimes. The study will be focusing on modelling the flow around 3D infinite foils and underwater glider swept hydrofoils to analyse transition to turbulence as well as the nature of the separation bubble. Development of a better understanding of these phenomena will help to achieve a more efficient design in the future.

%---------------------
\subsection{Characterisation of sheet cavity noise of a hydrofoil using the Ffowcs Williams-Hawkings acoustic analogy \cite{Lidtke2016}}

	Concerns about pollution of the marine environment with ship-induced noise and the scarcity of available numerical methods have recently stimulated significant amounts of research in hydroacoustic modelling. In this work, Large Eddy Simulation (LES) is used with Schnerr–Sauer mass-transfer cavitation model and a porous Ffowcs Williams–Hawkings (FW-H) acoustic analogy in order to simulate sheet cavitation on a NACA0009 hydrofoil. The aim is to investigate how well the proposed method captures the dominant noise sources associated with periodic sheet cavitation. The study further focuses on practical aspects, such as the importance of the non-linear FW-H term and convergence of the acoustic solution depending on the choice of the integration surface. This is done by correlating the radiated noise with integral and local flow quantities, such as cavity volume, lift coefficient and local vapour content. A key finding of the study is that the simulation framework is capable of correctly capturing the monopole nature of the sound generated by an oscillating cavity sheet. Results indicate that the numerical method is incapable of accurately resolving the flow during the final collapse stages of smaller cavity clouds, mainly due to mesh density limitations and the use of an incompressible flow assumption. Lack of small-scale bubble structures also causes the high-frequency range of the noise spectra to be under-predicted. Despite certain limitations the presented method offers a significant insight into the nature of cavitation-dominated noise and allows for some of the dominant sound generating mechanisms to be categorised.

%---------------------
\subsection{Multi-scale modelling of cavitation-induced pressure around the delft twist 11 hydrofoil \cite{Lidtke2016b}}

	A hybrid Lagrangian-Eulerian cavitation model based on the Schnerr-Sauer mass-transfer formulation is developed and then applied to study the flow around the Delft Twist 11 hydrofoil. The model uses volume-of-fluid approach to resolve large cavities and uses an interface reconstruction algorithm to identify vapour structures smaller than a grid-related threshold. These are then transferred to a Lagrangian framework and convected as particles act- ing as point noise sources. The underlying volume-of-fluid (VOF) model is shown to be in qualitatively good agreement with the experiment although it is found to under-predict the extent of cavitation. The combined model shows a substantial improvement in the prediction of near-field pressure fluctuations by accounting for the broadband contribution of bubbles smaller than the Eulerian grid size. In the pressure fluctuation spectra this is seen as a plateau extending to over a kilohertz beyond the low-frequency harmonics associated with the shedding frequency.

%---------------------
\subsection{Saving ocean soundscapes \cite{Lidtke2016c}}

	Current legislative climate for the introduction of rules governing noise emitted from commercial ships is explored, highlighting that it is not unimaginable that in the recent future designers and classification societies will need to concern themselves with predicting the noise radiated from marine propellers. A range of methods for accomplishing this is presented, ranging from full- and model-scale experiments and computational approaches. Relative strengths and weaknesses of all are discussed with the view of how they may fit into the ship design cycle in the future.

%---------------------
\subsection{Feasibility study into a computational approach for marine propeller noise and cavitation modelling \cite{Lidtke2016a}}

	There is increased interest in the ability to predict the noise associated with commercial ship propellers. Key components of the computational analysis process are considered for two test cases and the future direction in resolving the associated challenges is presented. Firstly, the Potsdam Propeller Test Case is used to compute tonal blade passage noise using the Ffowcs Williams-Hawkings acoustic analogy. Cavitation extents predicted using the Sauer and Schnerr mass transfer model agree well with the experiment but show little unsteadiness due to URANS being used. A complementary study of initial results from the study of cavitation noise modelling attempt is presented for a NACA0009 section, used as a simplified representation of a propeller blade. Large Eddy Simulation and FW-H acoustic analogy are used in order to estimate the cavitation-induced noise. Results indicate that the discussed approach provides the means for identifying low-frequency noise generation mechanisms in the flow, but does not allow for the fine-scale bubble dynamics or shockwave formation to be resolved. It is concluded that the discussed approach is a viable option to predict large parts of the marine propeller noise spectra but still further work is needed in order to account for the broadband components.

%---------------------
\subsection{RANS computations of flow around a bulk carrier with energy saving device \cite{Lidtke2015b}}

	The Fluid Structure Interactions group (University of Southampton) has been extensively involved in many research projects focusing on computations of ship wake field and the interactions between the propeller, rudder and the hull. Notable examples are works by Phillips et al., (2009), Banks et al., (2010), Badoe et al., (2013) and Winden et al., (2014). A finite-volume RANS code, OpenFOAM (OpenFOAM, 2014), has been used in majority of these works. The goal of the group has been to improve the in-house capability of prediction of ship stern flows using open-source software. In the present work OpenFOAM is benchmarked against a commercial code, Star-CCM+, with the aim of exploring the differences in flow field results originating from particular features of both implementations. The Japan Bulk Carrier (JBC) has been chosen as a test case representative of challenges faced in modern ship flow modelling. This vessel is fitted with an energy saving duct and as such presents an exciting case due to the complex flow field involved. The JBC case is part of the Tokyo 2015 CFD workshop and the latest in the series of benchmarking workshops to assess the state of art of marine CFD (Larsson et al., 2010). All computations are performed under steady state, fixed (even keel) conditions using identical grids and similar numerical setup. Presented analysis focuses on the mean flow, vortical structures and global hull forces.

%---------------------
\subsection{Using the FW-H equation for hydroacoustics of propellers \cite{Lloyd2015a}}

	Marine propeller noise and vibration has been studied in order to reduce acoustic signature, improve fatigue life and comfort onboard, and, more recently, to quantify the impact of maritime noise on the en vironment. Therefore the importance of acoustic measurements and simulations has increased. Acoustic simulations offer the advantages of avoiding vibration effects on measurement data, as well as the free placement of probes. These are not without difficulty however, and require robust verification, especially in the maritime field, where hydroacoustic computations are relatively uncommon. An attractive approach for making hydroacoustic predictions is an acoustic analogy, such as that of Ffowcs Williams and Hawkings (1969). The processes of sound generation and propagation are then conveniently separated, and acoustic predictions may be made as a post-processing of a hydrodynamic simulation. Since the precise location of the acoustic sources, such as cavitation and non-linear wake flow, are not known a priori in maritime applications (unlike in aeronautical applications, where, at low Mach number, they are assumed coincident with the blade surfaces), the porous formulation of the Ffowcs Williams-Hawkings (FW-H) equation is typically preferred (Di Francescantonio, 1997). In this case, the propeller is surrounded by a porous data surface (PDS) upon which velocity and pressure fluctuations representing the acoustic sources are extracted from a computational fluid dynamics (CFD) simulation. Uncertainty remains, however, regarding how best to use this method in practice. This is motivated by research showing that the propeller tip vortex may be a non-negligible noise source for non-cavitating conditions (Ianniello et al., 2013; Felli et al., 2015). Ianniello et al. (2013) and Lloyd et al. (2015) used an open-ended PDS to avoid the wake penetrating the downstream end cap, but did not investigate the effect of this. While similar studies have been performed in the aeroacoustics community for jet flows e.g. see Rahier et al. (2004), general experience of using such methods for marine propellers is not widespread (Ianniello and De Bernardis, 2015; Lloyd et al., 2014; Lidtke et al., 2015). In this paper we detail a study into appropriate use of the FW-H acoustic analogy for making hydroacoustic predictions of marine propellers. Two CFD codes (ReFRESCO and OpenFOAM) are used in order to provide additional verification. We focus on issues from the literature resulting from the simulation of a realistic propeller flow (e.g. Ianniello and De Bernardis, 2015), addressing two main questions, namely: how does closure of the PDS in the propeller wake affect the acoustic pressure prediction?; and what is the effect of the axial extent of the PDS in the downstream direction?

%---------------------
\subsection{Global Marine Technology Trends (GMTT) 2030 \cite{Shenoi2015}}

	Global Marine Technology Trends 2030 is the culmination of a collaborative project between Lloyd’s Register, QinetiQ and the University of Southampton. The report, released on Monday 7th September 2015, examines the transformative impact of eighteen technologies on ship design, on naval power and on the use of ocean space in 2030.

%---------------------
\subsection{Use of acoustic analogy for marine propeller noise characterisation \cite{Lidtke2015a}}

	Being able to predict shipborne noise is of significant importance to international maritime community. Porous Ffowcs-Williams Hawkings acoustic analogy is used with cavitation model by Sauer \& Schnerr in order to predict the noise signature of the Potsdam Propeller operating in open water. The radiation pattern is shown to be predominantly affected by a dipole source, in addition to less prominent sources at the propeller plane and in the wake. It is shown that the predicted sound pressure levels depend on the choice of the control surface and grid density. The unsteady RANS method is shown to be capable of capturing the blade harmonic noise components but lacks the ability to deal with the broadband part of the noise spectrum, both cavitation and turbulence induced, if no additional modelling is used.

%---------------------
\subsection{Comparison of various approaches to numerical simulation of ship resistance and propulsion \cite{Badoe2014}}

	The operation of a marine propeller dominates the flow interaction effects and alters the resistance on an upstream hull and the forces on a downstream rudder. A study is carried out into how these effects can be resolved by comparing four different methods. A classical prescribed body force approach in which an averaged nominal wake is used as input for the propeller model with prescribed thrust and torque; Two coupled BEMt-RANS solver which accounts for the non-uniform inflow into the propeller and a time resolved discretize propeller approach employing the use of an Arbitrary Mesh Interface model (AMI). The main differences between these four methods are also outlined quantitatively. The accurate results obtained using the two coupled BEMt-RANS approaches makes them fast and robust methods which can be used for ship resistance and self-propulsion estimation in the initial design phase.

%---------------------
\subsection{Outlook on marine propeller noise and cavitation modelling \cite{Lidtke2014a}}

	Two computational studies are presented in this paper. First, the Potsdam Propeller Test Case which is used to demonstrate the capabilities of mass transfer cavitation models, more precisely the model by Sauer and Schnerr, in tackling the problem of marine propeller cavitation. It is shown that the extents of the predicted cavitation regions agree well with the experiment but suffer from the fact that the tip vortices and the associated low pressure regions are under resolved when URANS is utilised. Next, preliminary results from the study of cavitation noise modelling attempt are presented for a NACA 0009 section, used as a simplified representation of a propeller blade. Large Eddy Simulation and Ffowcs Williams-Hawkings porous acoustic analogy are used in order to estimate the cavitation-induced noise. Results indicate that the discussed approach provides the means for identifying low-frequency noise generation mechanisms in the flow, yielding sound pressure levels of the order of 40 dB re 20 $\mu$Pa, but does not allow for finescale bubble dynamics to be resolved. One may conclude that the discussed approach is a viable option to predict large parts of marine propeller noise spectra but further work is needed in order to account for the high frequency components.

%---------------------
\subsection{The influence of turbulence modelling techniques on the predicted cavitation behaviour on a NACA0009 foil \cite{Lidtke2014}}

	The work presented here forms part of a project focusing on the development of cost-effective measures of classifying the noise levels from ship propellers with the use of numerical techniques available in OpenFOAM software. It is also related to the on-going research within the Faculty of Engineering and the Environment at the University of Southampton, looking at underwater noise of tidal turbines. Ultimately, the aim of the complete study is to enable the assessment of the environmental impact of a ship on the marine ecosystems. In this work a set of results from numerical experiments applied to the NACA0009 foil is presented in the context of quantifying the noise levels produced by a cavitating body in a uniform flow. The simulations utilise both URANS and LES methods and provide a means of characterising the differences between the observed flow patterns from the cavitation modelling point of view. In particular, the interactions of the cavitation phenomena with the turbulence, both modelled and resolved, are studied. Furthermore, an overview of how the considered cavitation models may be used for the purpose of noise prediction is given.

%---------------------
\subsection{Development of an America's Cup 45 tacking simulator \cite{Lidtke2013}}

	This paper describes the development of an AC45 simulator conducted as a student Master’s project at the University of Southampton. The main aim was to be able to asses and improve the tacking skills of the helm and the crew through systematic training. The physical interface of the simulator replicates the seating position of the helmsman and the main trimmer and the graphical representation provides the users with visual cues of the simulated boat, boundaries and marks for a sample race course. The theoretical model uses hydrodynamic manoeuvring coefficients based on empirical formulae and experimental data. The aerodynamic forces are pre-calculated using a full-scale RANS CFD simulation. The accuracy of the model is verified against the AC45 racing tracking data to ensure that the speed loss during a tack, experienced by the users of the simulator, is as close to reality as possible. The ultimate aim of the project was to study the potential of the simulator to assess and train the crews, improving their skill in tacking the boat effectively. This has been done by examining the performance of two groups of users over a series of practice sessions. The simulator could be potentially used for training the helmsmen of the Youth America’s Cup Red-Bull teams, which have limited budgets, training days and sailing experience compared to the professional AC sailors.


%%% References
\clearpage
{
\bibliography{../myPublications}{}
\bibliographystyle{ieeetr}
}


%----------------------------------------------------------------------------------------
\end{document}
